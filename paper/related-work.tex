%\section{Related Work}
\label{sec:related-work}

Domain decomposition (DD) techniques in general rely on the idea of splitting a larger domain of interest into smaller partitions or subdomains, which results in coupled Degrees-of-Freedom (DoF) at their common interfaces. Typical applications of DD in Boundary-Value problems (BVP) \cite{smith-ddm, lions-asm} have been successfully employed to efficiently compute the solution of large, discretized Partial Differential Equations (PDEs) in a scalable manner. 
DD techniques for parallel approximation of scattered data have been explored previously with Radial Basis Functions (RBF) \cite{mai-approx-rbf}, yielding good scalability and closely recovering the underlying solution profiles. In general, overlapping multiplicative and additive Schwarz \cite{orasm-as-ms-2007} iterative techniques for RBF \cite{ddm-rbf} have proven successful to tackle large-scale problems. Additionally, the use of restricted variants of additive Schwarz (RAS) method as preconditioners, with Krylov iterative solvers, can yield iterative schemes \cite{yokota-rasm-rbf} with $O(N)$ computational complexity, as opposed to the typical $O(N log(N))$ complexity with traditional RBF reconstructions \cite{ddm-rbf-fast}. The extensions of these ideas to B-spline bases exposes a way to fully parallelize traditional, serial MFA computations.

Combining the application of DD schemes and NURBS bases with isogeometric analysis (IGA) \cite{cottrell2009, da2012} for high-fidelity modeling of nonlinear PDEs \cite{dede2015, marini2015parallel, petiga-dalcin-2016} have enjoyed recent success at scale. However, many of these implementations lack full support to handle multiple geometric patches in a distributed memory setting due to non-trivial requirements on continuity constraints at patch boundaries. 
%Note that when using NURBS bases in a multipatch setting, the B-splines on the patch boundaries are interpolatory, thereby ensuring $C^0$ continuity in the solution for free. 
Directly imposing higher-order geometric continuity in IGA requires specialized parameterizations in order to preserve the approximation properties \cite{kapl2018construction}, which can be difficult to parallelize \cite{hofer2018fast} generally. 
%
In a similar vein, using B-spline bases to compute the MFA in parallel, while maintaining higher-order continuity across subdomains has not been fully explored previously. 

To overcome some of these issues with discontinuities along NURBS or B-spline patches, Zhang et al. \cite{zhang-nurbs-continuity} proposed to use a gradient projection scheme to constrain the value ($C^0$), the gradient ($C^1$), and the Hessian ($C^2$) at a small number of test points for optimal shape recovery. Such a constrained projection yields coupled systems of equations for control point data for local patches, and results in a global minimization problem that needs to be solved.

Alternatively, it is possible to create a constrained recovery during the actual post-processing stage i.e., during the decoding stage of the MFA through standard blending techniques \cite{grindeanu-blending}, in order to recover continuity in the decoded data. However, the underlying MFA representation remains discontinuous, and would become more so with increasing number of subdomains without the ability to recover higher-order derivatives along these boundaries. Moreover, selecting the amount of overlaps and resulting width of the blending region relies strongly on a heuristic, which can be problematic for general problem settings.

In contrast, we propose extensions to the constrained solvers used by Zhang et al. \cite{zhang-nurbs-continuity} and Xu et al. \cite{xu-jahn-discrete-adjoint}, and introduce a two-level, DD-based, parallel iterative scheme to enforce the true degree of continuity, independent of the basis function polynomial degree $p$, unlike the low-order constraints used previously  \cite{zhang-nurbs-continuity}. The outer iteration utilizes the RAS method \cite{gander-rasm}, with efficient inner subdomain solvers that can handle linear Least-Squares systems to minimize the decoded residual within acceptable error tolerances. %The inner subdomain solves can utilize adaptive MFA computations as well \cite{nashed-rational} with knot insertions and deletions to recover better reconstructions. 
Such an iterative solver has low memory requirements that scales weakly with growing number of subdomains, and necessitates only nearest-neighbor communication of the interface data once per outer iteration to converge towards consistent MFA solutions.
