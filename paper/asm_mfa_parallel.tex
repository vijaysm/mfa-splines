\documentclass[conference]{IEEEtran}
\IEEEoverridecommandlockouts
% The preceding line is only needed to identify funding in the first footnote. If that is unneeded, please comment it out.
\usepackage{cite}
\usepackage{amsmath,amssymb,amsfonts}
\usepackage{array}
\usepackage[ruled,noline,linesnumbered,noend]{algorithm2e}
\usepackage{graphicx}
%\usepackage{caption,subcaption,subfloat}
\usepackage[caption=false, font=footnotesize]{subfig}

\usepackage{textcomp}
\usepackage{xcolor}

% algorithm styling
\SetInd{0.5em}{0.2em}       % reduce indent
\newcommand\mycommentfont[1]{\footnotesize\sffamily{#1}}
\SetCommentSty{mycommentfont}
\SetKwRepeat{Do}{do}{while}%

% fun with colors
\RequirePackage{color}
\usepackage{colortbl}
\definecolor{RED}{rgb}{1,0,0}
\definecolor{BLUE}{rgb}{0,0,1}
\definecolor{GREEN}{rgb}{0,1,0}
\definecolor{color1}{rgb}{0.913, 0.776, 0.686}
\definecolor{color2}{rgb}{0.913, 0.867, 0.686}
\definecolor{ltgray}{rgb}{0.85, 0.85, 0.85}

% notes, remarks, todo
\newcommand{\Remark}[1]{{\color{RED}\sf Remark: {#1}}}
\newcommand{\tp}[1]{{\color{RED}\sf TP: {#1}}}
\newcommand{\dm}[1]{{\color{BLUE}\sf DM: {#1}}}
\newcommand{\Fix}[1]            {\textcolor{red}{\small\sf [#1]}}
\newcommand{\kw}[1]             {{\tt #1}\xspace}
\newcommand{\todo}[1]{
      \addcontentsline{tdo}{todo}{\protect{#1}}
      \marginpar{\colorbox{white!90!black}{\textcolor{red}{
      \parbox{2.1cm}{\scriptsize\bf\raggedright #1}
      }}}
}

\newcommand{\eqt}[1]{Equation~(\ref{#1})}
\newcommand{\fig}[1]{Fig.~(\ref{#1})}
\newcommand{\sect}[1]{Section~(\ref{#1})}
\newcommand{\algo}[1]{Algorithm~(\ref{#1})}

\usepackage{etoolbox}

\usepackage{wrapfig}
\usepackage{pgfplots}
\usepackage{tikz}
\usetikzlibrary{arrows}
\usetikzlibrary{decorations.shapes}
%\pgfplotsset{width=10cm,compat=1.9}
\usetikzlibrary{arrows.meta}
\usetikzlibrary{calc}

\tikzset{
	myarrow/.style={-{Triangle[length=3mm,width=1mm]}}
}

\usepackage[norndcorners,customcolors,nofill]{hf-tikz}
\hfsetbordercolor{black!50}


\def\BibTeX{{\rm B\kern-.05em{\sc i\kern-.025em b}\kern-.08em
    T\kern-.1667em\lower.7ex\hbox{E}\kern-.125emX}}
\begin{document}

\title{Parallel Domain Decomposition Techniques Applied to Multivariate Functional Approximation of Discrete Data\\
%{\footnotesize \textsuperscript{*}Note: Sub-titles are not captured in Xplore and should not be used}
%\thanks{Identify applicable funding agency here. If none, delete this.}
\thanks{Early Career Research Program, Department of Energy, US}
}

\author{\IEEEauthorblockN{1\textsuperscript{st} Vijay S. Mahadevan}
\IEEEauthorblockA{\textit{Mathematics and Computational Science Division} \\
\textit{Argonne National Laboratory}\\
Lemont, IL, 60439, USA \\
mahadevan@anl.gov}
\and
\IEEEauthorblockN{2\textsuperscript{nd} Thomas Peterka}
\IEEEauthorblockA{\textit{Mathematics and Computational Science Division} \\
\textit{Argonne National Laboratory}\\
Lemont, IL, 60439, USA \\
tpeterka@mcs.anl.gov}
\and
\IEEEauthorblockN{3\textsuperscript{rd} Iulian Grindeanu}
\IEEEauthorblockA{\textit{Mathematics and Computational Science Division} \\
\textit{Argonne National Laboratory}\\
Lemont, IL, 60439, USA \\
iulian@anl.gov}
\and
\IEEEauthorblockN{4\textsuperscript{th} Youssef Nashed}
\IEEEauthorblockA{\textit{Stats Perform}\\
Chicago, IL, 60601, USA \\
youssef.nashed@statsperform.com}
}

\maketitle

\begin{abstract}
Compactly expressing large-scale datasets through Multivariate Functional Approximations (MFA) can be critically important for analysis and visualization to drive scientific discovery. This paper presents scalable domain partitioning approach to compute MFA representation, by reducing the total work per task in combination with a nonlinear Schwarz-type, inner-outer iterative scheme for converging the interface data. For the underlying MFA, we utilize a tensor expansion of non-uniform B-spline (NURBS) basis in multi-dimensions to adaptively reduce the functional approximation error in the input data. While previous work on adaptive NURBS-based MFA has proven successful, the computational complexity of encoding large datasets on a single process can be severely prohibitive. Parallel algorithms for encoding domain-decomposed subdomains have had to rely on post-processing techniques to blend discontinuities across subdomains boundaries \cite{grindeanu-blending}. In contrast, this paper presents a robust domain constrained parallel solution infrastructure to impose higher-order continuity directly on the MFA representation computed.
We demonstrate effectiveness of the presented approach with an overlapping Restricted-type Additive Schwarz method (RASM) based domain decomposition solver, with a nonlinear accelerator such as $\ell$-BFGS or Krylov (CGS, L-GMRes) to minimize the subdomain error residuals of the decoded MFA, and more specifically to recover continuity across non-matching boundaries. The analysis of the presented scheme for analytical and scientific datasets in 1-D and 2-D are also presented. Additionally, scalability studies are also shown for scientific 2-D datasets from a Climate problem to evaluate the parallel speedup of the algorithm on large computing clusters.
\end{abstract}

\begin{IEEEkeywords}
functional approximation, domain decomposition, scalable methods, RASM
\end{IEEEkeywords}

\section{Introduction}

Large-scale discrete data analysis of various scientific computational simulations often require high-order continuous functional representations that have to be evaluated anywhere in the domain. Such expansions described as Multivariate Functional Approximations (MFA) \cite{de1983approximation} in arbitrary dimensions allow the original discrete data to be compressed, and expressed in a compact closed form in addition to supporting higher-order derivative queries. One particular option is to use NURBS bases \cite{nurbs-book} for the MFA encoding of scientific data. Due to the potentially large datasets that need to be encoded into MFA, the need for computationally efficient algorithms (in both time and memory) to parallelize the work is critically important. 

In the current paper, we utilize domain decomposition (DD) techniques \cite{smith-ddm} with data partitioning strategies to produce scalable algorithms to adaptively compute the MFA to reproduce a given dataset within user-specified tolerances. In such partitions, it is imperative to ensure that the continuity of the data across subdomain interfaces is maintained and is consistent with the degree of the underlying bases used in the NURBS-based MFA \cite{peterka-mfa}. 
We present an iterative DD scheme with an outer Schwarz-type iterative scheme in order to ensure that continuity is enforced, and the overall error stays bounded when number of subdomains is increased (subdomain size decreases).

\Remark{mention that method not limited to G0, G1 or G2 but arbitrary order upto p-1, where p is degree of NURBS bases}
%{\color{red}THIS IS A DRAFT}

%\begin{itemize}
%	\item Talk about MFA and how it can be used to approximation discrete solution data. Reference previous work.
%	\item Provide motivations on why this is necessary especially for large datasets
%	\item Literature survey of other work for parallel interpolation and compression of data
%	\item What are the other approaches to address this issue; pros and cons
%\end{itemize}

The paper is organized as follows. Section 2 summarizes the related work in using variations of the Schwarz scheme for scalable interpolation of data, and using constraints for recovering continuity along discontinuous patches. Section 3 provides details about the constrained optimization problem to resolve subdomain boundary discontinuities, along with the outer-inner DD-based solver setup to compute the continuous MFA. Next, the DD solver is applied to 1-D and 2-D analytical problems to verify error convergence, and scalability of the hierarchical scheme with decreasing subdomain sizes. The parallel scalability of the scheme is presented for scientific use-cases to adaptively compute the MFA within user-specified tolerances.


\section{Related Work}

DD techniques for parallel approximation of scattered data has been explored previously with Radial Basis Functions (RBF) \cite{mai-approx-rbf}, yielding good scalability and closely recovering the underlying profile. Overlapping multiplicative and additive Schwarz \cite{orasm-as-ms-2007} iterative techniques for RBF \cite{ddm-rbf} have proven successful to tackle large-scale problems. Additionally, the use of Restricted variants of Additive-Schwarz (RAS) method as preconditioners, with Krylov iterative solvers, has been shown to be scalable \cite{yokota-rasm-rbf} with $O(N)$ computational complexity, as opposed to the typical $O(N log(N))$ complexity with traditional RBF reconstructions \cite{ddm-rbf-fast}. 

Application of DD schemes and NURBS bases with isogeometric analysis (IGA) to high-fidelity modeling of nonlinear Partial Differential Equations (PDEs) have enjoyed recent success \cite{marini2015parallel, petiga-dalcin-2016} at scale, but many implementations lack full support to handle multiple geometric patches in a distributed memory setting due to non-trivial requirements on continuity constraints at patch boundaries. Note that when using NURBS bases in a multipatch setting, the B-splines on the patch boundaries are interpolatory, thereby ensuring $G_0$ continuity in the solution for free. However, incorporating higher order geometric continuity require specialized parameterizations in order to preserve the approximation properties \cite{kapl2018construction} and can be difficult to parallelize \cite{hofer2018fast}. 

However, to our knowledge, using a NURBS bases to compute the MFA in parallel, while maintaining higher-order continuity across subdomains, has not been explored previously. 
To overcome some of these issues with discontinuities along NURBS patches, Zhange et al \cite{zhang-nurbs-continuity} proposed to use a gradient projection scheme to constrain the value ($G_0$), the gradient ($G_1$), and the Hessian ($G_2$) at a small number of test points for optimal shape recovery. Such a constrained projection yields coupled systems of equations for control point data for local patches, and result in a global minimization problem that needs to be computed.

Alternatively, it is possible to create a constrained recovery during the actual post-processing stage i.e., during the decoding stage of the MFA through blending techniques \cite{grindeanu-blending}, in order to recover continuity in the decoded data. However, the underlying MFA representation remains discontinuous, and would become more so with increasing number of subdomains. Moreover, selecting the amount of overlaps and resulting width of the blending region relies strongly on a heuristic, which can be problematic.

In contrast, we propose extensions to the constrained solvers used by Zhang et al \cite{zhang-nurbs-continuity} and Xu et al \cite{xu-jahn-discrete-adjoint} by utilizing a recursive DD-based parallel outer-inner iterative scheme to enforce the degree of continuity prescribed by the user. The outer iteration utilizes RASM \cite{gander-rasm}, with an efficient, inner subdomain solver using $\ell$-BFGS as used in \cite{zheng-bo-bspline-bfgs}, or Krylov-type schemes with CGS, L-GMRes solvers, to minimize the decoded residual within acceptable error tolerances. This DD solver has low memory requirements that scales with growing number of subdomains, and imposes only nearest-neighbor communication of the interface data once per outer iteration. 

\section{Approach}

Domain decomposition techniques in general rely on the idea of splitting a larger domain of interest into smaller partitions or subdomains, which results in coupled Degrees-of-Freedom (DoF) at their common interfaces. Typical applications of DD in Boundary-Value problems (BVP) \cite{smith-ddm, lions-asm} have been successfully employed to efficiently compute the solution of large, discretized PDEs in a scalable manner. In the current work, we utilize a data decomposition approach, with extensions to overlap subdomain data to create shared layers in order to ensure that higher-order continuity across domain boundaries are preserved. This is essential to generate consistent and accurate MFA representations in parallel. Extending the overlapping Schwarz solvers for PDE applications to MFA computation in data analysis, the amount of overlap in the data for MFA can directly affect the global convergence speed of the iterative scheme, and hence the scalability of the overall algorithm \cite{bjorstad-overlap-1989}. %And the overall accuracy of the parallel algorithm cannot be worse than the single subdomain case.

In this section, we first provide an illustrative example by formulating the constrained minimization problem to be solved and explain the hierarchic iterative methodology used in the current work. We also describe the inner subdomain solvers that are used to compute optimal NURBS-based MFA representations in parallel in order to maintain higher-order continuity across subdomain boundaries.


%\begin{itemize}
%	\item What are we proposing and why this can be a stable technique to recover high-order continuity in parallel ?
%	\item Give context about DD methods and how ASM in this context makes sense 
%	\item Refer to \cite{smith-ddm} and \cite{ddm-rbf} as well and write out the equations with \cite{nurbs-book} help
%\end{itemize}

\subsection{Solver Methodology}
\label{sec:solver-methodology}

For the purpose of illustration and to explain the proposed solver methodology, consider a 1-D domain ($\Omega$) with two subdomains ($n_s=2$) as shown in \fig{fig:DD-subdomain-illiustration}, where $\Omega_1$ and $\Omega_2$ represent the subdomains that share an interface $d\Omega_{1,2}$. For generality, we also introduce an overlap layer $\Delta_1$ and $\Delta_2$ on each subdomain that share the decomposed data with its adjacent domain. 

\tikzset{decorate sep/.style 2 args=
	{decorate,decoration={shape backgrounds,shape=circle,shape size=#1,shape sep=#2}}}

\begin{figure}
	\centering
	\begin{tikzpicture}
	\begin{scope}[very thick]
	\begin{axis}[cycle list name=exotic, legend style={draw=none}, axis lines=none, xtick=\empty, ytick=\empty, xmin=0, xmax=100, ymin=0.8, ymax=1.2]
	%\begin{axis}[legend style={draw=none},xmin=-5, xmax=105, ymin=0.8, ymax=1.2]
	
	\addplot[domain=0:50, samples=6, color=blue, mark=halfcircle] {1.0};
	
	\addplot[domain=50:100, samples=6, color=red, mark=halfcircle] {1.0};
	%\addplot[color=blue] coordinates {0,0.5} ;
	%\addplot +[mark=none] coordinates {(0.5,1.25) (0.5,0.75)};
	%\addplot[domain=0.5:1, samples=6, color=black] {1};
	%\addplot coordinates {(0,1) (0.1,1) (0.2,1) (0.3,1) (0.4,1) (0.5,1) (0.6,1) (0.7,1) (0.8,1) (0.9,1) (1,1)};
	%\draw[decorate sep={1mm}{2mm},fill] (0,0) -- (100,0);
	%\draw[myarrow] (0,0) -- (1,5.5);
	
	\draw [decorate,decoration={brace,mirror,amplitude=20pt},color=blue,xshift=0pt,yshift=70pt]
	(0,0.9) -- (50,0.9) node [black,midway,yshift=-1cm] 
	{\footnotesize $\Omega_1$};
	
	\draw [decorate,decoration={brace,mirror,amplitude=5pt},color=blue,xshift=0pt,yshift=70pt]
	(50,0.9) -- (60,0.9) node [black,midway,yshift=-5mm] 
	{\footnotesize $\Delta_1$};
	
	\draw [decorate,decoration={brace,amplitude=20pt},color=red,xshift=0pt,yshift=90pt]
	(50,1.1) -- (100,1.1) node [black,midway,yshift=1cm] 
	{\footnotesize $\Omega_2$};
	
	\draw [decorate,decoration={brace,amplitude=5pt},color=red,xshift=0pt,yshift=90pt]
	(40,2.1) -- (50,2.1) node [black,midway,yshift=5mm] 
	{\footnotesize $\Delta_2$};
	
	\addplot [dashed, black, thick] coordinates {(50,1.15)  (50,0.85)} node[above left] {\footnotesize $\Omega_{1,2}$ } ;
	
	\end{axis}
	\end{scope}
	\end{tikzpicture}
	\caption{1-D parallel partitioned domain}
	\label{fig:DD-subdomain-illiustration}
\end{figure}


A $p$-th degree NURBS curve \cite{nurbs-book} is defined using the Cox-deBoor functions for each subdomain $i$ as

\begin{eqnarray}
C(u) &=& \sum_{i=0}^{n} R_{i,p}(u) P(i), \quad \forall u \in \Omega \\
R_{i,p}(u) &=& \frac{N_{i,p}(u) W_i}{\sum_{i=0}^{n} N_{i,p}(u) W_i}
\label{eq:nurbs-basis}
\end{eqnarray}

where $R_{i,p}(u)$ are the piecewise rational functions with $P$ control points, $W_i$ weights, and $p$-th degree B-spline bases $N_{i,p}(u)$ defined on a knot-vector $U$. Exact high-order derivatives of these NURBS basis defined in \eqt{eq:nurbs-basis} can also be evaluated without any approximation errors at the control point locations using the Cox-deBoor recurrence relations \cite{de1983approximation}.

Given a set of input points $Q$ that need to be encoded into a MFA, with the weights $W=1$ for simplicity, the minimization problem to compute the optimal set of control point locations within a subdomain can be posed as a solution to a linear Least-SQuares (LSQ) system. The LSQ solver computes optimal solution $P$ to minimize the objective $\left\lVert Q - R P \right\rVert$ \cite{nurbs-book}, which is the local subdomain residual error between the given input data and decoded MFA representation.

%\begin{equation}
%(N_i^T R_i) P_i = N_i^T Q_i, \forall i \in [1, \ldots n_s]
%\label{eq:LSQ-system}
%\end{equation}

In the current work, we use the unconstrained LSQ solver using Cholesky decomposition as the method of choice to compute the control point DoFs, when adaptively resolving the features in the input data through knot insertion and removal \cite{li-adaptive-2005}. Once the local subdomain resolution is sufficiently within user-specified tolerance levels, the resulting global MFA representation is piecewise discontinuous at subdomain boundaries. The next step is then to apply global constrained solvers to minimize the continuity error in order to recover higher derivatives iteratively as needed.

The global constrained minimization problem for the two subdomain case shown in \fig{fig:DD-subdomain-illiustration} can be written as
%
\begin{equation}
% A(X) X = F, \quad X = \left[P1 ; P2 \right]
\left[
\begin{array}{c|c}
R_{1} & \lambda_{1,2} \\
\hline
\lambda_{2,1} & R_{2}
\end{array}
\right]
\left[
\begin{array}{c}
P_{1} \\
P_{2}
\end{array}
\right]
=
\left[
\begin{array}{c}
Q_{1} \\
Q_{2}
\end{array}
\right]
\label{eq:global-system}
\end{equation}

%where
%\begin{eqnarray}
%A(X) &=&
%\left[
%\begin{array}{c|c}
%A_{1,1}(P_1) & A_{1,2}(P_1,P_2^*) \\
%\hline
%A_{2,1}(P_1^*,P_2) & A_{2,2}(P_1)
%\end{array}
%\right]
%\left[
%\begin{array}{c}
%P_{1} \\
%P_{2}
%\end{array}
%\right]
%, \\ 
%&\quad& \nonumber \\
%F &=& \left[
%\begin{array}{c}
%N_1^T Q_{1} \\
%N_2^T Q_{2}
%\end{array}
%\right]
%\label{eq:coupled-operator}
%\end{eqnarray}

The diagonal operators $R_{1}$ and $R_{2}$ are the piecewise rational functions that minimize the local subdomain residuals, while the off-diagonal blocks $\lambda_{1,2}$ and $\lambda_{2,1}$ represent the coupling terms between the subdomains at $\Omega_{1,2}$. This coupling term provides the constraints on the shared control point data, and higher-order derivatives as needed to recover smoothness and enforce continuity along subdomain boundaries. 

The coupling blocks $\lambda_{i,j}$ can be seen as Lagrange multipliers that explicitly couple the control point DoFs across a subdomain interface such that continuity is preserved in a weak sense \cite{nurbs-book}. We apply the RAS scheme to tackle this system of global equations shown in \eqt{eq:global-system}, where the coupled terms $\lambda_{i,j}$ utilize \textit{lagged} control point data from adjacent subdomains. It is easy to see that through a process of Gaussian elimination to remove the coupled operator dependency in each subdomain, the resulting set of constrained nonlinear equations can be solved iteratively. In a continuous sense, this can also be viewed as a block-Jacobi solver applied to \eqt{eq:global-system}.
%In the current paper, we instead use the Schur complement of \eqt{eq:coupled-operator} to eliminate the coupling terms by evaluating it at the lagged iterate values, in order to impose constraints in each subdomain independently. 

In this scheme, the coupled data $P_2, W_2$ and $P_1, W_1$ for subdomains $\Omega_1$ and $\Omega_2$ respectively at $\Omega_{1,2}$ are exchanged simultaneously before the local domain solves are computed. Since the exchanged constraint data is lagged at the previous iterate, the convergence rate in comparison to the more expensive multiplicative Schwarz variants \cite{smith-ddm} is slower. However, the key advantage to using RAS in the current methodology is that it only requires nearest neighbor exchange of data, that keeps communication costs bounded as number of subdomains increase \cite{orasm-as-ms-2007, gander-rasm}, while interlacing recomputation of the constrained control point solution. Note that in a RAS outer iterative scheme, nearest neighbor exchanges can be performed compactly per dimension and direction, thereby minimizing communication costs and eliminating global collectives.
%
%Once the interface constraint data terms are received, a Schur complement of \eqt{eq:coupled-operator} can be computed to eliminate the coupling terms, and to impose constraints in each subdomain independently. This leads to an augmented subdomain solution as shown in \eqt{eq:augmented-solution} that includes the control point, and optionally the derivative and weights at the interface obtained from the adjacent subdomain. 
%
%It is imperative to note that the only exchange of data between subdomains is performed through nearest neighbor communication that has bounded complexity and costs. 
The volume of messages exchanged however depends on several factors.

\begin{enumerate}
	\item \textbf{Continuity}: The degree of continuity determines the stencil needed to enforce the constraints on either side of the interface
	\item \textbf{Overlap}: The amount of overlap ($\Delta$) determines the number of coupled data layers to be communicated between neighboring domains
\end{enumerate}

%
%\begin{equation}
%%\tikzset{left offset=-0.1,right offset=0.02,disable rounded corners=true}
%\tikzset{offset def/.style={
%		above left offset={-0.1,0.8},
%		below right offset={0.1,-0.65},
%	}
%}
%%\begin{array}{@{} *{2}{ c @{} >{{}}c<{{}} @{} } c @{}}
%%16\cdot \tikzmarkin{A}I_1  & + & 11\cdot \tikzmarkin{B}I_2  & = & \tikzmarkin{C}13\\[1ex]
%%11\cdot I_1\tikzmarkend{A} & + & 16\cdot I_2\tikzmarkend{B} & = & 17\tikzmarkend{C}
%%\end{array}
%P_1^{'} =
%\left[
%\begin{array}{c}
%\tikzmarkin{A}(1,-0.3)(-0.1,0.3) P_1(1)   \\
%P_1 (2) \\
%\vdots   \\
%\tikzmarkend{A} P_1 (m) \\
%$\quad$ \vspace*{-2mm} \\
%\tikzmarkin{B}(0.6,-0.1)(-0.1,0.3) P_2(1) \\
%0 \\
%\vdots \\
%\tikzmarkend{B} 0
%\end{array}
%\right],
%P_2^{'} =
%\left[
%\begin{array}{c}
%\tikzmarkin{C}(1,-0.3)(-0.1,0.3) P_2(1)   \\
%P_2 (2) \\
%\vdots   \\
%\tikzmarkend{C} P_2 (n) \\
%$\quad$ \vspace*{-2mm} \\
%\tikzmarkin{D}(0.6,-0.1)(-0.1,0.3) P_1(1) \\
%0 \\
%\vdots \\
%\tikzmarkend{D} 0
%\end{array}
%\right]
%\label{eq:augmented-solution}
%\end{equation}
%%
%\begin{tikzpicture}[remember picture,overlay]
%\pgfsetarrowsend{latex} 
%%
%% adjust the shift from "col" to move the position of the annotation
%\coordinate (A-aa) at ($(A)+(-0.5,-1.0)$);
%\node[align=left,left] at (A-aa) {\footnotesize{$P_1$($\Omega_1$)}};
%\path[>=stealth,blue,draw] (A-aa) -- ($(A)+(0.2,-1.0)$);
%%
%% adjust the shift from "col" to move the position of the annotation
%\coordinate (B-aa) at ($(B)+(-0.5,-1.0)$);
%\node[align=left,left] at (B-aa) {\footnotesize{$P_1$($\Delta_1$)}};
%\path[>=stealth,blue,draw] (B-aa) -- ($(B)+(0.2,-1.0)$);
%%
%% adjust the shift from "col" to move the position of the annotation
%\coordinate (C-aa) at ($(C)+(1.4,-1.0)$);
%\node[align=right,right] at (C-aa) {\footnotesize{$P_2$($\Omega_2$)}};
%\path[>=stealth,red,draw] (C-aa) -- ($(C)+(0.75,-1.0)$);
%%
%% adjust the shift from "col" to move the position of the annotation
%\coordinate (D-aa) at ($(D)+(1.4,-1.0)$);
%\node[align=right,right] at (D-aa) {\footnotesize{$P_2$($\Delta_2$)}};
%\path[>=stealth,red,draw] (D-aa) -- ($(D)+(0.75,-1.0)$);
%%
%\end{tikzpicture}
%%
%where $m, n$ are the number of control points in $\Omega_1$ and $\Omega_2$ respectively.

At convergence, the interface data at $\Omega_{1,2}$ will satisfy the higher-order continuity prescriptions specified by the user ($G_0$ to $G_{p-1}$). The illustration and description can be generalized and extended to arbitrary dimensions and will be used as the basis to describe the local subdomain solvers in the following subsections.

%\subsection{Constrained Linear Least-Squares Solver}
%
%Given the user specification to obtain a $G_0, G_1$ or $G_2$ continuity across block interfaces, the local subdomain solve can essentially be formulated as the solution to a LSQ problem with constraints. The idea here is to introduce additional unknowns in the solutions, which are essentially the Lagrange multipliers to obtain an augmented system of equations.
%
%If we consider the local system introduced in \eqt{eq:LSQ-system}, the solution $P$ is the unconstrained control point data that is only piecewise continuous globally. Then, we can introduce a constrained set of equations \cite{nurbs-book} such that 
%
%\begin{equation}
%(N_i^T M_i) P_i = N_i^T T_i, \forall i \in [1, \ldots n_s]
%\label{eq:LSQ-system-constraint}
%\end{equation}
%
%where $M$ is the basis matrix corresponding to the constrained DoFs and $T$ is the vector of input data in the overlap region of interest $\Delta$. Then the constrained linear least squares curve fitting problem is simply minimizing the residual $E=Q-RP$ such that $MP=T$ is satisfied. We refer the readers to Section (9.4.2) in \cite{nurbs-book} for further details on how a Schur complement of the coupled equation systems can be applied to eliminate the Lagrange multipliers in order to enforce the constraints within each subdomain system. This particular strategy can also be extended to higher dimensions easily with larger number of constraints included in the $M$ operator.

We next present two variations of the subdomain solvers that are used in this study. 

\subsection{Constrained Nonlinear Solver}

In order to ensure continuity across NURBS patches, Zhang et al  \cite{zhang-nurbs-continuity} evaluated the constraint matrix at test points along the interface curve. These computations require a Singular Value Decomposition (SVD) solve at every interface and can become prohibitively expensive as dimensionality increases. Alternatively, we can use a nonlinear minimizer, directly applied to the system in \eqt{eq:global-system}, and adding a penalized term for the constrained DoFs represented by the coupling blocks $\lambda_{i,j}$. Such a system can be expressed as a scalar minimization problem in each subdomain with the following objective functional.
%
\begin{eqnarray}
%r_1 &=& (Q_1 - R_1 P_1) + \epsilon \left[ (P_1(\Delta_1) - P_2(\Omega_2 \cap \Delta_1 )) \right], \forall P_1 \in \Omega_1  \\
%%
%r_2 &=& (Q_2 - R_2 P_2) + \epsilon \left[ (P_2(\Delta_2) - P_1(\Omega_1 \cap \Delta_2)) \right], \forall P_2 \in \Omega_2 
r_i(\Omega_i) &=& \left\lVert Q_i - R_i P_i \right\rVert_2, \quad \forall i \in [1, n_s] \nonumber \\
r_c(\Delta_{i})    &=& \sum_{j \in \partial \Omega_i} \left\lVert P_i(\Delta_i) - P_{j}(\Omega_{j} \cap \Delta_{i} )  \right\rVert_2 \nonumber \\ 
r_i(\Omega_i \bigcup \Delta_{i}) &=&  r_i(\Omega_i) + \epsilon r_c(\Delta_{i}), 
%&+& \epsilon \left[ (1-\delta_{i,0}) (P_i(\Delta_i) - P_{i+1}(\Omega_{i+1} \cap \Delta_{i} ))  \right]^2 \nonumber \\ 
%&+& \epsilon \left[ (1-\delta_{i,n_s}) (P_i(\Delta_{i}) - P_{i-1}(\Omega_{i-1} \cap \Delta_i)) \right]^2, \nonumber \\
%r_i^2 &=& \left\lVert Q_i - R_i P_i \right\rVert_2 \nonumber \\
%&+& \epsilon \left[ (1-\delta_{i,0}) (P_i(\Delta_i) - P_{i+1}(\Omega_{i+1} \cap \Delta_{i} ))  \right]^2 \nonumber \\ 
%&+& \epsilon \left[ (1-\delta_{i,n_s}) (P_i(\Delta_{i}) - P_{i-1}(\Omega_{i-1} \cap \Delta_i)) \right]^2, \nonumber 
\label{eq:nonlinear-residuals}
\end{eqnarray}

where $\epsilon$ is the boundary penalty term. In our experiments, a large value of about $10^{7}$ has yielded very good performance with favorably monotonic error reduction in the constraint residuals $r_c(\Delta_{i})$.

The augmented solution for each local subdomain after nearest neighbor exchange can contain control points and corresponding derivative data to constrain DoFs on both the left and right interfaces in 1-D. This definition extends naturally to higher dimensions, with the penalized constraints on the boundary terms evaluated as a loop over all shared interfaces to compute the net residual. 

To compute the solution to the minimization problem in \eqt{eq:nonlinear-residuals}, we can make use of Sequential Least SQuares Programming (SLSQP) solver with equality constraints, or apply the limited memory version of  Broyden-Fletcher-Goldfarb-Shannon ($\ell$-BFGS) algorithm when the dimensionality increases, and memory requirements become large. Such solvers have already been proven effective for converging such constrained continuity problems with B-Splines \cite{zheng-bo-bspline-bfgs}. A much more efficient method is to reformulate the objective function as a root finding problem, and applying Krylov accelerators or Anderson mixing to compute the optimally constrained solutions, with much lower computational complexity and superior convergence properties. The minimized control point solution is achieved when the interface solutions match on all $\Omega_{i,j} \in \Omega$ to drive the nonlinear residuals in \eqt{eq:nonlinear-residuals} to within solver relative tolerance of $10^{-12}$.

%A much more efficient solver would also be to use a Krylov accelerator for the subdomain solves, which expresses the components in \eqt{eq:nonlinear-residuals} as a vector to reformulate it as a root finding problem. In our experiments, the Krylov accelerator was typically much faster to converge and compute the optimally constrained solutions.

\subsection{Constrained Nonlinear Solver on Decoded Data}

Instead of imposing the constraints in the control point space, alternatively, we can utilize the expansion of the MFA in input point space directly to minimize the decoded residual $E= Q - R P$. There are some advantages to this approach compared to imposing the interface constraints in the control point space, even though the potential volume of data to be communicated between subdomains is much larger.

\Remark{this subsection still needs a bit of work}

\begin{enumerate}
	\item Subdomain residuals and boundary constraints are both in the decoded space, and hence no explicit need for a residual projection (encoding) or penalty term in \eqt{eq:nonlinear-residuals}
	\item No explicit need for imposition of higher order derivative constraints, especially in the context of non-conforming adaptivity across subdomain interfaces
	\item Natural extensions within the ASM iterative scheme to generate overlapping variants
\end{enumerate}

%Explain the iterative scheme in terms of the underlying equations and how the boundary terms are resolved through a global ASM method. First start with 1-D and talk about extensions in the scheme to allow arbitrary dimensional solver framework.

When the solver is setup to use the error residuals in the actual decoded space, the natural data decomposition with overlap $\Delta$ can yield good improvements in both time and accuracy \cite{bjorstad-overlap-1989}. This follows the effectiveness of the overlapping additive-Schwarz preconditioning schemes in the context of linear algebra problems for PDEs \cite{smith-ddm} \cite{gander-rasm}. 

%\begin{equation}
%(N_i^T M_i) P_i = N_i^T T_i, \forall i \in [1, \ldots n_s]
%\label{eq:decoded-residuals}
%\end{equation}


\section{Implementation}
\label{sec:implementation}

The presented methods in this manuscript have been primarily implemented in Python using bindings for the DIY C++ library. DIY~\cite{morozov16} is a programming model and runtime
for block-parallel analytics on distributed-memory machines, built on MPI-3~\cite{dongarra13}.  Rather than programming
for process parallelism directly in MPI, the programming model in DIY is based on block parallelism: data are decomposed
into subdomains called blocks; blocks are assigned to processing elements (processes or threads); computation is
described over these blocks, and communication between blocks is defined by reusable patterns. The same DIY program
consisting of a block-parallel decomposition can be run on different numbers of MPI processes: it is the job of the DIY
runtime to map between blocks and processes. The Python bindings to DIY, pyDIY~\cite{pydiy},
are a more recent development that utilize PyBind11~\cite{jakob17} and MPI4Py~\cite{dalcin11} to expose the interfaces in the C++ library. In our implementation, PyDIY and DIY exclusively manage the data decomposition, including specifications to share an interface $\Omega_{i,j}$ and ghost layers that represent overlaps $\Delta$.

The overall approach is sketched in \algo{alg:pseudocode}.
% \Remark{Vijay: finish the pseudocode and expand the discussion of it below.} 
We begin by decomposing the domain into a set of regular blocks aligned with the principal axes
of the global domain. Before enforcing constraints, the local subdomain solves are performed completely decoupled so that the discontinuous MFA to represent the partitioned input data is computed. We also include adaptive knot placements to approximate the input data by introducing control point DoFs where needed to capture solution data variations. The control point solution from this adaptive, decoupled LSQ problem solver is then used as the DoF data that needs to be constrained with RAS iterative method.
We then begin iterating over the blocks in a 2-level nested loop: the outer loop is driven by RAS
iterations described in \sect{sec:solver-methodology}, while the inner loop executes the individual subdomain MFA solves simultaneously to compute the control point solution to the nonlinear optimization problem in \eqt{eq:nonlinear-residuals}
% -- \ref{eq:decoded-residuals} 
with BFGS or Krylov methods. 
At the start of each outer ASM iteration, the control point constraints are exchanged between neighboring blocks in a regular nearest-neighbor communication pattern. This is sufficient to update the constraints $P(\Delta)$ for the inner subdomain solves. The DIY send and receive data exchange API enqueues and dequeues the constraint data to neighboring blocks based on the parallel data decomposition.
Depending on whether the MFA residual or the decoded residual norms are to be minimized, a subdomain solver is chosen to drive the nonlinear residual for the local block within user-specified tolerances.
At the end of the outer iterative loop, we check for convergence across all subdomains, by evaluating whether the maxima of the $L_{\infty}$ error across all subdomains ($n_s$) of the ASM update vector is within user-specified tolerance. If this condition is satisfied, the MFA computation is stopped.
On convergence, the result is a global MFA that retains high-order continuity and accuracy of a single subdomain solve, but with excellent parallel efficiency to reduce total time to solution as the number of subdomains increases.

%\Remark{\algo{alg:pseudocode} is just a skeleton. Details need to be filled in. Termination conditions, local (inner loop) iterative scheme, types of constraints (control points or decoded points), etc.}


\begin{algorithm}
	\DontPrintSemicolon
	decompose domain into blocks with DIY\;
	solve local MFA without constraints\;
	\tcp*[h]{Outer ASM loop;}\;
	iASM $\leftarrow$ 0 \tcp*{ASM iteration counter}
	\Do{ dPMax $ > 10^{-10}$ and iASM $<$ nMaxASM}
	{
		$P(\Omega \cap \Delta)$ $\rightarrow$ enqueue outgoing constraints\;
		exchange constraints with neighbor blocks\;
		$P(\Delta)$ $\leftarrow$ dequeue incoming constraints\;
		\tcp*[h]{Inner BFGS/Krylov optimization loop}\;
		\For {$isub \in n_s$}
		{
			$P_i \leftarrow$ solve adaptive local MFA with constraints\;
			$E_i \leftarrow$ compute decoded local error\;
			$\delta P (\Omega_i) \leftarrow P_{iASM} (\Omega_i) - P_{iASM-1} (\Omega_i)$\;
		}
		dPMax $\leftarrow \left\lVert \delta P (\Omega_i) \right\rVert_{\infty}$\;
		iASM++\;
	}
	\tcp*[h]{Store subdomain solution data;}\;
    Write MFA to disk for analysis and visualization
	\caption{Hierarchic DD MFA Solver}
	\label{alg:pseudocode}
\end{algorithm}


\section{Results}
\label{sec:results}

To demonstrate the effectiveness of the iterative algorithm for MFA computation, we devised a series of analytical closed form functionals and utilized real-world scientific datasets in both 2- and 3-dimensions obtained from high-fidelity simulations. All runs shown in this section were performed using the Python drivers written specifically for this work using the \texttt{DIY} domain decomposition infrastructure.

%A brief description of the problem cases that are used as test datasets is given below.
%
%\begin{enumerate}
%	\item Synthetic data: sinc($x$)=$\frac{sin(x)}{x}$ functional on $\Omega \in [-4, 4]$
%	\begin{itemize}
%		\item \textbf{Symmetric \dimension{1}:} Q(x) = 100(sinc($x$-1) + sinc($x$+1))
%		\item \textbf{Asymmetric \dimension{1}:} Q(x) = sinc($x$) + sinc(2$x$-1) + sinc(3$x$+1.5), 
%		% z = scale * ( np.sinc((X+1)**2 + (Y-1)**2) + np.sinc(((X-1)**2 + (Y+1)**2)) )
%		\item \textbf{Analytical 2d:} Q(x,y) = sinc($(x+1)^2+(y-1)^2$) + sinc($(x-1)^2+(y+1)^2$)
%	\end{itemize}
%%	\item Nek5000: A 3-D dataset from CFD simulation with velocity magnitude as the solution profile
%	\item S3D: A 3-D turbulent combustion dataset generated by an S3D simulation \cite{chen-s3d-2009} of fuel jet combustion in the presence of an external cross-flow. 
%	\item CESM: A 2-D Community Atmosphere Model climate model dataset on a sphere with 3600x1800 resolution
%\end{enumerate}


\subsection{\dimension{1} Results}\label{sec:results-1d}

In this section, detailed analysis on the convergence and accuracy of various MFA continuity recovery approaches are presented. 
%\begin{itemize}
%	\item Use the first two problems to measure convergence in parallel as number of domains increase
%	\item Talk about adaptivity and resolution of data even for highly varying problem data.
%\end{itemize}

\subsubsection{Comparison of Clamped vs Floating Boundary Knots}

To demonstrate the choice of using floating knots vs the low-order ($C^0$) continuous clamped knots at subdomain boundaries, we choose an analytical closed form reference solution of the form:

\begin{equation}
	F(x) = sinc(x) + sinc(2x-1) + sinc(3x+1.5), \forall x \in \Omega=[-4, 4]
	\label{eqn:1d-asymmetric-sinc}
\end{equation}

The reference solution $F(x)$, the results from the clamped knots, and floating knots with and without augmented overlap regions $\delta=p$ are shown in \fig{fig:comparison-clamped-floating}. The figures show the recovered solutions and the corresponding decoded error from MFA evaluation for a $\mathcal{N}=2$ and $p=3$ case. It is evident that the net error profile in the fully clamped subdivision in this example shows lower error as compared to the floating knot experiments. However, it is imperative to note that the former only shows $C^0$ regularity, while the floating knots fully recover high-order continuity at subdomain interfaces. Moreover, the use of augmented overlap regions ($\delta=3$) produce error profiles that resemble a single subdomain error profile in the domain, which is one of the key metrics of interest. These behaviors and conclusions extend to multi-dimensional setting as well.

\begin{figure}[htbp]
	\centering
	\subfloat[Input analytical \dimension{1} solution profile\label{fig:1d-reference}]{%
		\includegraphics[width=0.44\textwidth]{figures/1d-reference-solution}}
	\quad \quad
	\subfloat[Clamped $C^0$ continuous decoded solution\label{fig:1d-clamped}]{%
		\includegraphics[width=0.44\linewidth]{figures/1d-clamped-2subd}}
	\\
	\subfloat[Floating knots at $\Omega_{1,2}$ recovering $C^{p-1}$ continuity\label{fig:1d-unclamped}]{%
		\includegraphics[width=0.44\linewidth]{figures/1d-unclamped-2subd}}
	\quad \quad
	\subfloat[Floating knots at $\Omega_{1,2}$ with $\left| \delta \right|=3$ \label{fig:1d-unclamped-aug}]{%
		\includegraphics[width=0.44\linewidth]{figures/1d-unclamped-aug-2subd}}
	\caption{\dimension{1} analytical sinc dataset with 10000 input points with $\mathcal{N}=2$ and $p=3$}
	\label{fig:comparison-clamped-floating}
\end{figure}


%\comment{Discuss that the error in clamped solution are lower but only $C^0$, while in the floating version, we get $C^p$}

%\subsubsection{Error Convergence and Continuity Verification}

To further demonstrate the continuity recovery behavior, we plot the error profile $\vec{E}$ for these approaches in \fig{fig:1d-error-comparison}, zoomed in around the interface $\Omega_{1,2}$. The unconstrained and decoupled LSQ solution procedure in the top shows that the reconstructed solution is discontinuous at the interface, as expected. Using the fully clamped approach that yields lower overall absolute error showcases only $C^0$ continuity at the interface, which may or not be sufficient depending on the use case utilizing the MFA representation. Finally, the bottom plot shows the smooth error profile from using the floating knots at the interface with full recovery of high-order continuity. We again emphasize that one could recover $C^0$ to $C^{p-1}$ continuity with this approach by choosing to use floating knots vs varying number of repeated knots at the interface.

%The solution profile, along with control point locations from the adaptive computations are shown in \fig{fig:sinc-adaptive-a}. We utilized the residual definition shown in \eqt{eq:nonlinear-residuals} to ensure $C_2$ continuity (with $p=3$) to be satisfied along all internal subdomain interfaces. The \fig{fig:sinc-adaptive-b} and \fig{fig:sinc-adaptive-c} present the zoomed in reconstruction of the approximation near the interface between $\Omega_2$ and $\Omega_3$ before, and after the ASM iterations are converged. The initially discontinuous solutions from independent, adaptive, unconstrained subdomain solves resulting in \fig{fig:sinc-adaptive-b} are then iteratively converged to satisfy all constraints after 3 RAS iterations, with nearest neighbor exchange of control point data at the interfaces to recover $C_2$ continuity.

\begin{figure}[htbp]
	\centering
	\includegraphics[width=0.44\textwidth]{figures/1d-error-comparison}
	\caption{Zoomed error plots at interface $\Omega_{1,2}$ for \dimension{1} analytical dataset with $10^4$ input points, $\mathcal{N}=2$ and $p=3$.  Top: unconstrained and decoupled subdomain profile (discontinuous). Middle: clamped interface knots. Bottom: floating interface knots.}
	\label{fig:1d-error-comparison}
\end{figure}

%\subsubsection{Subdomain Solver Performance}
%
%Compare LCLSQ - Encoded nonlinear and Decoded nonlinear

\subsubsection{Error Convergence and Overlap Experiments}

To determine the effect of using augmented or overlapped knot span regions ($\delta$) as the number of subdomains $\mathcal{N}$ are increased, we use a fully symmetric double-sinc function as shown below in \eqt{eqn:1d-symmetric-sinc} on a single subdomain as the reference solution as shown in \fig{fig:error-1d-a}, and with $\mathcal{N}=5$ for different values of augmented spans ($\left| \delta \right|=0$ and $\left| \delta \right|=3$).


\begin{equation}
	F(x) = sinc(x+1) + sinc(x-1), \forall x \in \Omega=[-4, 4]
	\label{eqn:1d-symmetric-sinc}
\end{equation}


\begin{figure}[htbp]
	\centering
	\subfloat[A double sinc profile in \dimension{1}\label{fig:error-1d-a}]{%
		\includegraphics[width=0.5\textwidth]{figures/sol-subd1-cpts63-d3-g0.png}}
	\hfill
	\subfloat[Error profile for $\mathcal{N}=1$ and $p=3$\label{fig:error-1d-b}]{%
		\includegraphics[width=0.5\textwidth]{figures/subd1-cpts63-d3-g0.png}}
	\\
	\subfloat[Error profile for $\mathcal{N}=5$, $p=3$ and $\left| \delta \right|=0$\label{fig:error-1d-c}]{%
		\includegraphics[width=0.5\textwidth]{figures/subd5-cpts15-d3-g0.png}}
	\hfill
	\subfloat[Error profile for $\mathcal{N}=5$, $p=3$ and $\left| \delta \right|=p$\label{fig:error-1d-d}]{%
	\includegraphics[width=0.5\textwidth]{figures/subd5-cpts15-d3-g3.png}}
	\caption{Demonstration of error convergence, and effect of the overlapping spans to minimize numerical artifacts}
	\label{fig:error-1d}
\end{figure}

It is evident from \fig{fig:error-1d-c} that when there is no augmented knot spans used in a multi-subdomain solver, the decoding of data at subdomain boundaries are influenced by contributions from both adjacent domain DoFs, which are enforced to be $C^{p-1}$ continuous by the constrained minimization solver. However, as we increase the number of overlap regions in terms of both the underlying data and the local bases support spans, the error profiles as shown in \fig{fig:error-1d-d} approaches the reference profile (with $\mathcal{N}=1$) shown in \fig{fig:error-1d-b}. Heuristically, for many of the problems tested, using $\left| \delta \right|=p$ provides optimal convergence as number of subdomains increase, even though increasing this parameter to $\left| \delta \right| =2p$ or higher will in general always improve the numerical accuracy at the cost of higher communication costs between neighboring subdomains.

%\comment{Might be worthwhile to evaluate (first) derivatives and show continuity there as well}


\subsection{Multi-dimensional Problem Cases}

In this section, we present some results from applying parallel MFA for multidimensional problem cases.

\subsubsection{2D Problem}

A \dimension{3} vector field representing the numerical results of a large-eddy simulation of Navier-Stokes equations for the MAX experiment \cite{merzari2010numerical} is representative of turbulent mixing and thermal striping that occur in the upper plenum of liquid sodium fast reactors. The data, generated by the Nek5000 solver \cite{deville2002}, have been resampled from their original topology onto a $200 \times 200 \times 200$ regular grid, and the magnitude of the velocity vector is associated with each \dimension{3} domain point \cite{peterka-mfa}.
%
Out of this dataset, a 2D slice (with $\left| \Omega \right|=200\times 200$) along the midplane in axial direction is used for our first study here. The reference solution and the converged, reconstructed solution with $\mathcal{N}=5 \times 5=25$ subdomains with $p=6$ and $\left| \delta \right|=2p$ is shown in \fig{fig:nek2d-ref-reconstruct} for different compression ratios. Depending on the usecase for MFA reconstruction, the converged error norms with 20 floating control points per subdomain yielding a net compression of 4X is sufficient to evaluate continuous derivatives everywhere in the domain $\Omega$. A full MFA representation (with compression ratio=1) is also shown in \fig{fig:nek2d-d6-n1}, which can fully reconstruct the original features in the input dataset in contrast to \fig{fig:nek2d-d6-n5} that shows a lossy smoothing of the sharp features in the original data.

\begin{figure}[htbp]
	\centering
	\subfloat[Nek5000 2D slice: Reference\label{fig:nek2d-reference-a}]{%
		\includegraphics[width=0.33\textwidth]{figures/2d-nek-d6-n5-c20-aug6-reference.png}}
	\subfloat[Solution: 4x compression \label{fig:nek2d-d6-n5}]{%
		\includegraphics[width=0.33\textwidth]{figures/2d-nek-d6-n5-c20-aug6}}
	\subfloat[Solution: 1x compression\label{fig:nek2d-d6-n1}]{%
		\includegraphics[width=0.33\textwidth]{figures/2d-nek-d6-n1-c200}}
	\caption{2D slice of the Nek5000 \dimension{3} dataset ($200 \times 200$): reference profile and B-spline MFA with $p=6$, $\mathcal{N}=5\times5$, $\left| \delta \right|=p$ for 4X compression (middle), and 1X compression (right).}
	\label{fig:nek2d-ref-reconstruct}
\end{figure}


\subsubsection{\dimension{3} Problem}

Next, we present S3D, a turbulent fuel jet combustion dataset generated from a simulation in the presence of an external cross-flow \cite{chen2009}. The \dimension{3} domain has the span $\left| \Omega \right|=704 \times 540 \times 550$, with the raw data containing three components of the vector field. We choose to use the magnitude of this velocity field in our reconstruction study shown below in \fig{fig:s3d-ref-error} with 209M points. The converged MFA reconstruction shown in the figure with $8^3=512$ subdomains and $n=35$ per direction  in each subdomain yields a net compression ratio of $\frac{(704\times540\times550)}{(8\times35)^3} \approx 9.5$. While uniform refinement in knot spans does yield sufficient error reductions in most subdomains, utilizing adaptive error resolution with knot insertions and removals for MFA as previously used here for single subdomains \cite{nashed-rational} for the S3D problem can provide better reconstructions in addition to the iterative scheme introduced here. Such extensions will be pursued in the future. However, the experiments demonstrate that the MFA computations produce reconstructed data and  numerical errors that are consistent and convergent for arbitrary $\mathcal{N}, n, p$ and values of $\left| \delta \right|$, proving the feasibility of the algorithm.

\begin{figure}[htbp]
	\centering
	\subfloat[S3D dataset profile\label{fig:s3d-refprofile}]{%
		\includegraphics[width=0.33\textwidth]{figures/s3d-volume-profile-reference3.png}}
	\subfloat[Converged solution profile\label{fig:s3d-profile}]{%
			\includegraphics[width=0.33\textwidth]{figures/s3d-volume-profile-conv3.png}}
	\subfloat[Final error profile\label{fig:s3d-error}]{%
	\includegraphics[width=0.33\textwidth]{figures/s3d-volume-profile-error-conv3.png}}
	\caption{Volume rendered S3D dataset with compression rate $\eta=9.5$: reference profile, converged MFA decoded profile and the corresponding reconstruction error with $\mathcal{N}=8\times 8\times 8=512$, $p=3$ and  $\left| \delta \right|=2p=6$}
	\label{fig:s3d-ref-error}
\end{figure}


\subsubsection{Error Convergence}

%\comment{Show error convergence in \dimension{3} problem to look at how iteratively things resolve}

We utilize synthetic datasets shown in \eqt{eqt:2d-synthetic} and \eqt{eqt:3d-synthetic} to perform error convergence studies and to determine areas of maximal error that iteratively are resolved between neighboring subdomains.

%\vspace*{-4mm}
%
%\begin{equation}
%	F(x,y)  = sinc(\sqrt{x^2+y^2}) + sinc(2(x - 2)^2 + 2(y + 2)^2), \forall (x,y) \in [-4,4]^2
%	\label{eqt:2d-synthetic}
%\end{equation}
%
%\vspace*{-8mm}
%
%\begin{equation}
%	F(x,y,z) = sinc(\sqrt{x^2 + y^2 + z^2}) + sinc(2(x - 2)^2 + (y + 2)^2 + (z - 2)^2), \forall  (x,y,z) \in [-4,4]^d.
%	%\\
%	%\quad \quad &\forall& \Omega = (x,y,z) \in [-4,4]^d.
%	\label{eqt:3d-synthetic}
%\end{equation}


\begin{eqnarray}
	F(x,y)  &=& sinc(\sqrt{x^2+y^2}) + sinc(2(x - 2)^2 + 2(y + 2)^2), \forall (x,y) \in [-4,4]^2, \label{eqt:2d-synthetic}\\
	\quad \quad F(x,y,z) &=& sinc(\sqrt{x^2 + y^2 + z^2}) + sinc(2(x - 2)^2 + (y + 2)^2 + (z - 2)^2), \forall  (x,y,z) \in [-4,4]^d.
	\label{eqt:3d-synthetic}
\end{eqnarray}

We plot the change in error $\vec{E}$ between subsequent iterations of the RAS scheme for both \dimension{2} and \dimension{3} problem cases with $\left| \delta \right|=0$ in \fig{fig:2d-3d-error-convergence}. This clearly demonstrates that the $\mathcal{SS}$ interface values are resolved first, and then $\mathcal{MS}$ DoFs are resolved further. In all cases, the iterations converge in 2 steps, independent of $\mathcal{N}$ or $\delta$.
%
\begin{figure}[htbp]
	\centering
	\subfloat[\dimension{2} problem with $\mathcal{N}=5\times 5$: Iteration 1\label{fig:2d-iter1}]{%
		\includegraphics[width=0.44\textwidth]{figures/2d-d3-n5-aug0-iter1-threshold0001.png}}
	\quad \quad
	\subfloat[\dimension{2} problem with $\mathcal{N}=5\times 5$: Iteration 2\label{fig:2d-iter2}]{%
		\includegraphics[width=0.44\textwidth]{figures/2d-d3-n5-aug0-iter2-threshold0001.png}}\\
	\subfloat[\dimension{3} problem with $\mathcal{N}=3\times 3 \times 3$: Iteration 1\label{fig:3d-iter1}]{%
	\includegraphics[width=0.44\textwidth]{figures/3d-d5-n4-aug0-iter1-threshold01.png}}
	\quad \quad
	\subfloat[\dimension{3} problem with $\mathcal{N}=3\times 3 \times 3$: Iteration 2\label{fig:3d-iter2}]{%
	\includegraphics[width=0.44\textwidth]{figures/3d-d5-n4-aug0-iter2-threshold001.png}}\\
	\caption{Error convergence for \dimension{2} and \dimension{3} problems on the first and second iterate}
	\label{fig:2d-3d-error-convergence}
\end{figure}



%Using the 2-D sinc analytical function, the error convergence for different NURBS bases degree $p$ was measured. As the number of control point DoFs are increased, the global norm of the errors are reduced consistently in the current implementation. Next we tested the spatial adaptivity in 2-D on a scientific dataset to ensure that the hierarchical iterative scheme with RAS is able to capture strong solution variations, and still ensure continuity across subdomain interfaces. 
%Using varying fidelity of the real world data, and the ability to create complex closed form functionals within our Python implementation, the approach was verification for accuracy using uniform control point refinements to yield sizes theoretical convergence orders. 
%We utilize the described 2-D problem datasets to better understand the convergence and parallel scalability of the proposed algorithms at scale. Demonstrations for the S3D, and CESM datasets are provided below.

%\subsubsection{S3D Dataset: Spatial Adaptivity}

%Additionally, using adaptive resolution based on a-posteriori gradient estimates, numerical errors in strongly varying solution profiles were reduced to within user-specified tolerances. A sample solution for the 2-D slice of the S3D dataset on 9 subdomains is shown in \fig{fig:s3d-adaptive-2d}, along with the error profiles at the first iteration in \fig{fig:s3d-adaptive-2d-b} and on convergence of the RAS iterations in {fig:s3d-adaptive-2d-c}.
%
%\begin{figure}[htbp]
%	\centering
%	\subfloat[S3D dataset profile\label{fig:s3d-adaptive-2d-a}]{%
%	\includegraphics[width=0.44\textwidth]{figures/s3d-2d-profile.png}}
%	\hfill
%	\subfloat[Error profile before RAS iteration or adaptivity\label{fig:s3d-adaptive-2d-b}]{%
%		\includegraphics[width=0.44\textwidth]{figures/s3d-error-iter1}}\\
%	\subfloat[Error profile after RAS iteration converges\label{fig:s3d-adaptive-2d-c}]{%
%		\includegraphics[width=0.44\textwidth]{figures/s3d-error-iter4}}
%	\caption{\comment{2-D slice of the S3D dataset: profile and adaptive error resolution with tolerance=$10^{-2}$}}
%	\label{fig:s3d-adaptive-2d}
%\end{figure}

%The errors in the decoupled LSQ solve shown in \fig{fig:s3d-adaptive-2d-b} has relatively larger errors in the subdomains with strong gradient changes. When including knot adaptivity combined with RAS iterative scheme, the errors in the reconstructed MFA data are reduced to within user-specified relative tolerances of $10^{-2}$. This result demonstrates the convergence of the proposed scheme to adaptively resolve a complex scientific solution profile, even on a small number of subdomains ($\mathcal{N}=16$). 

%\Remark{show error convergence plots for S3D on 16 subdomains with adaptivity}

%that even with the coarse resolution with 16 subdomains, 

%\subsubsection{Nek5000 dataset}
%
%\Remark{add table of actual compression and error as $n_s$ is changed}

%\Remark{Discuss about complications and potential ways to enforce continuity. (a) Use decoded data, (b) Use control point space across interface}



\subsubsection{Parallel Scalability}\label{sec:parallel-scalability}

To demonstrate the parallel performance of the implemented RAS iterative scheme for MFA computation with continuity preservation, we employed both closed-form synthetic and real simulation datasets in \dimension{2} and \dimension{3}. In the following sections, we present both strong scaling and weak scaling studies performed on the Theta Cray XC40 supercomputer operated by the Argonne Leadership Computing Facility (ALCF), which provides 4,392 KNL compute nodes with 64 compute cores and 192 GB DDR4 RAM per node. The interconnect is based on the Aries Dragonfly high speed network.

\subsubsection*{Strong Scaling}

We consider both \dimension{2} and \dimension{3} problems to demonstrate the scaling behavior of the presented parallel MFA computational algorithm in \algo{alg:pseudocode}. One key consideration that drove selection of the subdomain sizes, and the floating knot span descriptions, is motivated by the metric to recover the original error profile from a single subdomain case. 
Verification studies were performed during this strong scaling test to ensure that the local subdomain errors computed on a single task, and on different process counts remain the same at convergence. This verification is important to reiterate the fact that the approximation error due to the constrained solves to recover higher-order continuity does not significantly affect the error metrics for the MFA as $\mathcal{N}$ increases. For this reason, we used synthetic datasets generated with closed form equations for \dimension{2} and \dimension{3} studies as shown in \eqt{eqt:2d-synthetic} and \eqt{eqt:3d-synthetic}.

%The parallel scalability of the implemented RAS iterative solver for ensuring continuity across block boundaries was measured on the 2-D CESM dataset, whose solution profile in a uniform latitude-longitude grid is shown in \fig{fig:cesm-2d-profile}. 
%Our experiments were executed on the Bebop cluster at
%Argonne National Laboratory LCRC (Laboratory Computing Resource Center). Bebop has 1024 public nodes, with Intel Broadwell or Intel Knights Landing processors, with 128 GB memory per Broadwell node, and 104 GB per KNL node. It has an Omni-Path Fabric Interconnect. Broadwell processors have 36 cores, while KNL processors have 64 cores. We primarily used the KNL processors for the scalability tests presented below. \comment{Actually we used haswell nodes here}. 

%\comment{Also mention that the scaling results below were run on Theta ALCF on KNL nodes}

The strong scaling tests were performed on 1 to 16,384 parallel tasks in \dimension{2}, increasing by a factor of $2^2=4$, and the \dimension{3} tests were executed on 1 to 32,768 tasks, increasing by a factor of $2^3=8$. We also note that the \dimension{2} and \dimension{3} case for these studies used 400M and 1.331B input points respectively, with a corresponding $\eta=4$ and $\eta=1.25$ that was maintained constant for all runs. 
In order to also better understand the effects of using augmented overlap regions ($\delta$) on scalability, two cases with the choice of $\left| \delta \right|=0$ and $\left| \delta \right|=p$ are shown in \fig{fig:strong-scaling}. 
%
%A strong scalability test was performed on the Bebop cluster with the CESM dataset, using 32 subdomains in the X-direction and 128 subdomains in the Y-direction. With this fixed DD and a uniformly placed, 100 control points per subdomain (10x10) resolution, parallel jobs were executed to compute the MFA. 
%
The Python driver utilized \texttt{DIY} to handle block decompositions and rank assignments, as the total number of tasks used in the parallel run was increased. We measured the overall computational time for setting up the problem, the initial subdomain solves, and the consequent RAS iteration cycle to convergence, which includes the nearest neighbor communication at each iteration. We also show the time for decoding the MFA that is used to measure the errors in each subdomain, and the overall total that includes the effort spent on each of these various tasks. This task-wise breakdown helps us clearly visualize the steps that scale linearly and the ones that do not.

%The time to compute the MFA in parallel is shown for this strong scalability experiment in \fig{fig:cesm-strong-scalability}.

%\begin{figure}[htbp]
%	\includegraphics[width=0.45\textwidth]{figures/cesm-profile.png}
%	\caption{2-D CESM climate dataset}
%	\label{fig:cesm-2d-profile}
%\end{figure}



\begin{figure}[htbp]
	\centering
	\subfloat[\dimension{2} strong scaling\label{fig:strong-scaling-2d-a}]{%
		\includegraphics[width=0.49\textwidth]{figures/2d-strong-scaling.png}}
	\subfloat[\dimension{2} strong scaling with $\left| \delta \right|=p$\label{fig:strong-scaling-2d-b}]{%
		\includegraphics[width=0.49\textwidth]{figures/2d-strong-scaling-aug.png}}\\
	\subfloat[\dimension{3} strong scaling\label{fig:strong-scaling-3d-a}]{%
		\includegraphics[width=0.49\textwidth]{figures/3d-strong-scaling.png}}
	\subfloat[\dimension{3} strong scaling with $\left| \delta \right|=p$\label{fig:strong-scaling-3d-b}]{%
		\includegraphics[width=0.49\textwidth]{figures/3d-strong-scaling-aug.png}}\\
	\caption{Strong scaling performance of RAS solver with $p=3$}
	\label{fig:strong-scaling}
\end{figure}


As expected, the RAS iterative scheme shows excellent scalability for the chosen datasets, and the overall time to compute the MFA in parallel was reduced at a nearly ideal rate up to $10^3$ MPI tasks as $\mathcal{N}$ increases, while ensuring $C^{p-1}$ continuity in the subdomain interfaces.  It is important to note that the dominant computational time is driven by the decoupled LSQ solution computation and decoding operations, which are embarassingly parallel as the size of the subdomains (determined by number of knot spans) decreases in direct proportion to the tasks. Given that the scalability of the linear algebraic LSQ solvers \cite{benzi2002, dalcin2011} and Sparse Matrix Vector (SpMV) products used in the decode tasks are well understood, the bottlenecks potentially occur only from the nearest neighbor communication for constraint data exchanges, which remain relatively small in magnitude up to 32K processes tested in the \dimension{3} experiments. The overall strong scaling efficiency remains around 50\% for both overlapping and non-overlapping \dimension{2} problem cases at 16,384 tasks. However, the added setup cost and less than ideally scaling nearest neighbor communications reduce the \dimension{3} problem efficiency for the overlapping subdomain cases to 25\% at 32,768 tasks from 70\% in the nonoverlapping cases. 

Additionally, the parallel efficiency degradation behavior in augmented runs due to the high setup cost occurs due to the current choice of implementations to determine intervals in both the input space data ($\vec{Q}$) and extra knot spans ($\delta$) that need to be exchanged with neighboring subdomains. This setup cost is purely local in the case when $\left| \delta \right|=0$, in contrast to the communication dominated setup when $\left| \delta \right| > 0$. 
%\comment{talk about parallel efficiency values}
It is evident that the cost of nearest neighbor exchanges at scale is driven by the size of the messages being transferred between subdomains, which is a function of $\left| \delta \right|$. The increased amount of data shows smaller degradation in communication for \dimension{2} problems as compared to \dimension{3}, and hence the strong scaling behavior is dependent on the dimension of the problem being solved that is representative of performance in all structured grid solvers for PDEs.

%The ratio of computational work in local subdomain solves in comparison with communication time in nearest-neighbor exchanges gradually increases as subdomain size shrinks. This is due to the fact that global information on smaller subdomains take multiple iterations to propagate, and hence using overlaps for such large problems would be a recommended extension in the future to improve overall scalability of the algorithm. For this intermediate size problem, the strong scaling efficiency of the presented scheme is around 35\% on 2,048 processes, which dropped off from 65\% parallel efficiency at 256 processes.


%\Remark{should we try to do a performance model for this particular case ?
%	communication: C = 4096*4 = 16384 messages
%	serial = 5C + 5A
%	parallel = 5C + 5A/p
%}

%CESM dataset has 6.5M data points
%We used 32*128 = 4096 subdomains for the strong scaling test
%\Remark{How does the nearest neighbor communication stay bounded ? measure timings separately ?}

%jobrun_neup cesm_p1 1 1 python ../Projection-2D.py -x 32 -y 128 -p 5 -c 10 -a 5 --disableadaptivity 

%\begin{figure}
%	\begin{tikzpicture}
%	
%	\begin{loglogaxis}[
%	legend pos= north east,
%%	ymin = 0.0005, ymax = 0.01,
%	xlabel=Number of Processes,
%	ylabel=Time to Compute MFA (secs),
%%	nodes near coords*={\Label},
%%	visualization depends on={value \thisrow{label} \as \Label},
%	grid=both,
%	grid style={line width=.1pt, draw=gray!10},
%	major grid style={line width=.2pt,draw=gray!50},
%	axis line style={latex-latex},
%	every axis plot/.append style={line width=1.0pt},
%	%axis lines=middle,
%	]
%	\axispath\draw
%	(7.49165,-10.02171)
%	|-  (8.31801,-11.32467)
%	node[near start,left] {$\frac{dy}{dx} = -2.58$};
%	
%	\addplot[smooth,mark=*,blue,width=5pt] table [meta=eff]  {
%		x y eff
%		1	7045.451504	100
%		2	4137.296798	85.14558959
%		4	2322.884537	75.82653583
%		8	1061.974002	82.92871918
%		16	544.925238	80.80754722
%		32	314.578249	69.98906001
%		64	150.8316872	72.98544611
%		128	83.16858159	66.18195096
%		256	42.09792775	65.37446476
%		512	25.07828297	54.870772
%		1024	17.07229925	40.30109616
%		2048	9.898160849	34.75556641
%%		4096	8.519652482	20.18956686
%	};
%
%	\addplot[dashed,black,width=2pt] table [meta=class]  {
%	x y class 
%	1	40000	100
%	2	20000	100
%	4	10000	100
%	8	5000	100
%	16	2500	100
%	32	1250	100
%	64	625	100
%	128	312.5	100
%	256	156.25	100
%	512	78.125	100
%	1024	39.0625	100
%	2048	19.53125	100
%%	4096	9.765625	100
%};
%	
%	\legend{Scalability\\Ideal\\}
%	
%	\end{loglogaxis}
%	\end{tikzpicture}
%	\caption{Strong scalability of the 2-D CESM Problem with 4,096 subdomains and 100 control points/subdomain}
%	\label{fig:cesm-strong-scalability}
%\end{figure}




\subsubsection*{Weak Scaling}

Given that the performance of the overlapping and augmented MFA scheme was comparable to non-overlapping cases ($\left| \delta \right|=0$), and since the error reduction from having extra overlaps always results in better solution recovery, we strictly focus on overlapping cases alone for the weak scaling study. Here, the overall work per subdomain is maintained constant, and the number of tasks are increased from 1 to 16,384 in \dimension{2}, and from 1 to 32,768 in \dimension{3}, similar to the strong scaling study.
The weak scaling results maintaining an overall MFA compression rate of $\eta=2^d,\forall d \in [2,3]$ is shown in \fig{fig:weak-scaling}.


\begin{figure}[htbp]
	\centering
	\subfloat[\dimension{2} weak scaling\label{fig:weak-scaling-2d}]{%
		\includegraphics[width=0.5\textwidth]{figures/2d-weak-scaling.png}}
	\hfill
	\subfloat[\dimension{3} weak scaling\label{fig:weak-scaling-3d}]{%
		\includegraphics[width=0.5\textwidth]{figures/3d-weak-scaling.png}}\\
	\caption{Weak scaling performance of RAS solver with $p=3$ and $\left| \delta \right|=p$}
	\label{fig:weak-scaling}
\end{figure}

The weak scaling study demonstrates that the overall performance of the RAS iterative scheme for large number of subdomains does not significantly affect the parallel efficiency, which are around 40\% in \dimension{2} and 24\% in \dimension{3} at the fine limit tested. The subdomain solve and nearest neighbor data exchange dominate the overall time to solution. However, it is important to note that the actual runtime for the MFA computation only grows by a factor of 2, even on 16K processes or more. 


\subsubsection*{Performance Study on S3D Dataset}

Finally, we consider the case of the S3D combustion dataset shown in \fig{fig:s3d-ref-error} and measure the strong scaling performance on up to 4,096 processes. Using parallel MPI-IO implemented with \texttt{DIY}, and exposed through the Python interface, a strong scaling performance study was measured on this realistic dataset and shown in \fig{fig:s3d-strong-scaling}. Note that the IO cost for reading the chunk of data required on each task is included in the setup time shown in the figure.

The performance and error analysis indicate good speedup to reduce overall time for MFA computation, until nearest neighbor communication and data exchanges start dominating the overall workflow. These results show similar behavior to the strong scaling studies performed on synthetic datasets and provide confirmation on the feasibility of the presented approach for tackling real-world large datasets.

\begin{figure}[htbp]
	\centering
	\includegraphics[width=0.6\textwidth]{figures/problem-s3d-strong-scaling.png}
	\caption{Strong scaling for the S3D dataset with $p=5$}
	\label{fig:s3d-strong-scaling}
\end{figure}


%\paragraph{Positioning Figures and Tables} Place figures and tables at the top and 
%``Fig.~\ref{fig}'', even at the beginning of a sentence.

%\begin{table}[htbp]
%\caption{Table Type Styles}
%\begin{center}
%\begin{tabular}{|c|c|c|c|}
%\hline
%\textbf{Table}&\multicolumn{3}{|c|}{\textbf{Table Column Head}} \\
%\cline{2-4} 
%\textbf{Head} & \textbf{\textit{Table column subhead}}& \textbf{\textit{Subhead}}& \textbf{\textit{Subhead}} \\
%\hline
%copy& More table copy$^{\mathrm{a}}$& &  \\
%\hline
%\multicolumn{4}{l}{$^{\mathrm{a}}$Sample of a Table footnote.}
%\end{tabular}
%\label{tab1}
%\end{center}
%\end{table}

%\begin{figure}[htbp]
%\centerline{\includegraphics{fig1.png}}
%\caption{Example of a figure caption.}
%\label{fig}
%\end{figure}


\section{Conclusion}

We have presented a scalable DD approach to tackle the issue of discontinuous MFA representations when performing the computations in parallel. Restricted Additive Schwarz (RAS) method is a natural algorithmic fit for data analysis problems to create efficient MFA encoders in parallel. Through the user of Schwarz-based iterative schemes, combined with $\ell$-BFGS or Krylov solvers for local subdomain solves, the hierarchical iterative technique was proven to be robust in converging to the compressed functional representation of the given data, without sacrificing the approximation accuracy. Combining NURBS-based adaptivity with a-posteriori error measures, and ensuring higher-order continuity across block boundaries, a scalable infrastructure has been presented. The PyDIY based Python implementations for 1-D and 2-D problems have been shown here to resolve complex solution profiles and gradient variations, even under decreasing subdomain sizes. The use of overlap layers can definitely improve the overall MFA accuracy and convergence speed of the RAS algorithm as demonstrated for some 1-D analytical problems, but at a slightly higher cost per iteration. Further analysis on the effect of overlap regions for RAS in higher-dimensional settings needs to be explored. 

The strong scalability of the algorithm was also explored for a reasonably large 2-D climate dataset with 6.5M data points. The outer-inner iterative scheme provides very good strong scalability up to 1024 processes after which the performance degrades slightly due to the lack of local computation work in comparison to the latency and communication costs associated with the nearest-neighbor exchanges of constraint data. 

A more natural way to ensure continuity across NURBS patches would be to use T-splines \cite{sederberg-2004}, which is specifically designed for merging higher-dimensional surfaces with non-matching knot locations. The implementation of T-splines for adaptivity in the context of MFA is currently being explored, and the presented ASM based solver approach can still be used to impose constraints across subdomain patch boundaries, while local constraints within each block to satisfy hanging node DoFs can be imposed with appropriate T-spline basis modifications.

The methodology and experiments presented in the paper are implemented mostly in Python. The current Pythonic interface, PyDIY is only a barebones interface. These algorithms will be ported to the more production-ready C++ MFA codebase \cite{mfa-codebase} in the future. The new implementation will directly make use to the DIY C++ library and all its features for data decomposition along with fast local subdomain solvers without overheads typically associated with Python codes. 
%and hence, a better estimate of the overall performance of both the RAS iterative scheme and the subdomain solvers without the overheads of dependent solvers in Python will provide 

%{\color{red} 
%	it would also be interesting to use multilevel techniques to create a hierarchical MFA representation and use ASM to accelerate finer solves using coarser control point data. Need to explore this idea further but can propose it here as future exploration topic
%}

%\begin{itemize}
%	\item What did we implement to enhance speedup of the MFA framework and did we preserve accuracy of the underlying method ?
%	\item Did we speedup the actual computation by performing DD with ASM global iterations for some of the problem data ?
%	\item Does the method scale as a function of domains and problem size ? 
%	\item What advantages does it provide for fix-up schemes that can be used in a post-processing step (ref Iulian's blending idea) ?
%	\item Future extensions to T-splines and local adaptivity and potential complications involved
%\end{itemize}

\section*{Acknowledgment}

This work is supported by Advanced Scientific Computing Research, Office of Science, U.S. Department of Energy, under
Contract DE-AC02-06CH11357, program manager Laura Biven. We gratefully acknowledge the computing resources provided on
Bebop, a high-performance computing cluster operated by the Laboratory Computing Resource Center (LCRC) at Argonne
National Laboratory.

\bibliographystyle{IEEEtran}
\bibliography{asm-mfa}

%\vspace{12pt}
%\color{red}
%IEEE conference templates contain guidance text for composing and formatting conference papers. Please ensure that all template text is removed from your conference paper prior to submission to the conference. Failure to remove the template text from your paper may result in your paper not being published.

\end{document}
